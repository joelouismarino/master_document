\chapter{Intelligence}

\section{Intelligence}

Intelligence is a vaguely defined concept. In many situations, we often use the term to refer to the ability to perceive and understand certain concepts, such as math, art, politics, business, etc. But this narrow definition is human specific and fails to capture the broad spectrum that intelligence occupies. Instead, I use this working definition:

\begin{center}
	\textbf{Intelligence} is the ability of a system to perform meaningful actions within a particular environment.
\end{center}

We could argue whether a system that can perceive aspects of its environment but is unable to act is intelligent, but this is meaningless, as this system has no practical purpose. This definition has a few components: A \textbf{system} is some collection of matter: a molecular structure, single-celled organism, animal, machine, computer, human, society, etc. \textbf{Meaningful actions} are more difficult to define. In general terms, these are some non-random interactions with the environment, i.e. the distribution of actions given the environmental state has a relatively low entropy. Often we also associate these actions with achieving some goal or reward, although this is not strictly necessary. Finally, \textbf{within a particular environment} emphasizes the idea that intelligence is always specific to an environment (the data). Intelligence is not a characteristic of a system alone, it's always conditioned on the environment.

\section{Creating Intelligence}

There are two ways in which to create an intelligent system: through \textbf{design} and through \textbf{learning}. In design, one intelligent system constructs another intelligent system. In learning, a system gains intelligence through interaction with the environment.