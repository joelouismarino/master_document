\chapter{Intelligence}

\section{Intelligence}

Intelligence is often vaguely defined. In many situations, we use the term to refer to the ability to perceive and understand certain concepts, such as math, art, politics, business, etc., but this narrow definition fails to capture the spectrum that intelligence occupies. Instead, I use the following definition:

\begin{center}
	\textbf{Intelligence} is the ability of a system to interact meaningfully within an environment.
\end{center}

We could argue whether or not a passive system that can only \textit{perceive} aspects of its environment is intelligent, but this is fruitless, as such a system has no practical purpose. This definition has a few components: A \textbf{system} is some collection of matter: a molecular structure, single-celled organism, animal, machine, computer, human, society, etc. \textbf{Interact meaningfully}, in general terms, refers to non-random interactions with the environment, i.e. the distribution of actions given the environmental state has low entropy. \textit{todo: refine this definition.} These actions are typically associated with achieving some goal or reward, although this is not strictly necessary. Finally, \textbf{within an environment} emphasizes that intelligence is specific to an environment (the data). Intelligence is not a characteristic of a system in isolation; it is always conditioned on a particular context that the system is suited to handle.

\section{Creating Intelligence}

There are two ways in which to create an intelligent system: through \textbf{design} and through \textbf{learning}. In design, one intelligent system constructs another intelligent system. In learning, a system gains intelligence through interaction with the environment.