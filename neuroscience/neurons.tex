\chapter{Neurons}
\label{chap: neurons}

Neurons serve as a basic building block of all nervous systems. In computational neuroscience, they often serve as the basic computational unit, implementing the primitive set of computations. In this chapter, we will discuss the basic functioning of neurons, then we will explore the diversity of neurons, both in terms of anatomy and response properties.


Hodgkin-Huxley model \cite{hodgkin1952quantitative}, STDP \cite{markram1997regulation, bi1998synaptic}


\section{Synapses}

Gordon Shepherd refers to synapses as the ``elementary structural and functional unit for the construction of neural circuits" \cite{shepherd2003synaptic}. The functioning of a synapse can be generally characterized as follows:
\begin{enumerate}
    \item The presynaptic membrane depolarizes.
    \item Calcium (Ca$^{2+}$) ions enter the presynaptic terminal.
    \item Through a sequence of mechanisms, a synaptic vesicle fuses with the membrane, releasing a quantum of neurotransmitter into the \textit{synaptic cleft}.
    \item Neurotransmitter molecules diffuse across the cleft, eventually binding to receptor molecules in the postsynaptic membrane.
    \item The conductance of an ionotropic receptor is modified, changing the excitability of the postsynaptic region.
\end{enumerate}
If the postsynaptic region is depolarized (higher voltage), then it is referred to as an excitatory postsynaptic potential (EPSP). If the region is hyperpolarized (lower voltage), then it is referred to as an inhibitory postsynaptic potential (IPSP). However, in addition to depolarization and hyperpolarization, neurotransmitters can enact other, more complex changes. By activating metabotropic receptors, neurotransmitters can use second-messenger pathways underlying changes in the postsynaptic and presynaptic cell. Such pathways are important for activity-dependent effects like long-term potentiation (LTP) and long-term depression (LTD), thought to be involved in learning and memory. Thus, despite their size, synapses are intricate structures that can transmit various signals over multiple time scales.

Synapses can generally be divided into two groups depending on the densification of their presynaptic and postsynaptic membranes: asymmetrical (type 1) and symmetrical (type 2). Type 1 synapses have been associated with excitatory activity, whereas type 2 synapses have been associated with inhibitory activity. They also differ in the types of vesicles that they contain (round vs. flat).  


\section{Dendrites}




\section{Soma}


\section{Axon}


\section{Action Potentials}


\section{Catalog of Neurons}


\section{Activity Profiles}

